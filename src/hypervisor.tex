\chapter{Hypervisor Extension, Version 0.2-draft}
\label{hypervisor}

This chapter describes the RISC-V hypervisor extension, which virtualizes the
supervisor-level architecture to support the efficient hosting of guest
operating systems atop a type-1 or type-2 hypervisor.
The hypervisor extension changes supervisor mode into
{\em hypervisor-extended supervisor mode} (HS-mode, or {\em hypervisor
mode} for short), where a hypervisor or a hosting-capable operating system
runs.  The hypervisor extension also adds another level of address translation,
from {\em guest physical addresses} to supervisor physical addresses,
to virtualize the
memory and memory-mapped I/O subsystems for a guest operating system.  HS-mode
acts the same as S-mode, but with additional instructions and CSRs that control
the new level of address translation and support hosting a guest OS in virtual
S-mode (VS-mode).
Regular S-mode operating systems can execute without modification either in
HS-mode or as VS-mode guests.

In HS-mode, an OS or hypervisor interacts with the machine through the same
SBI as an OS normally does from S-mode.  An HS-mode hypervisor is expected to
implement the SBI for its VS-mode guest.

The hypervisor extension is enabled by setting bit 7 in the {\tt misa} CSR,
which corresponds to the letter H.  When {\tt misa}[7] is clear, the hart
behaves as though this extension were not implemented, and attempts to use
hypervisor CSRs or instructions raise an illegal instruction exception.
Implementations that include the hypervisor extension are encouraged
not to hardwire {\tt misa}[7], so that the extension may be disabled.

\begin{commentary}
This draft is based on earlier proposals by John Hauser and Paolo Bonzini.
\end{commentary}

\begin{commentary}
The baseline privileged architecture is designed to simplify the use of classic
virtualization techniques, where a guest OS is run at user-level, as
the few privileged instructions can be easily detected and trapped.
The hypervisor extension improves virtualization performance by
reducing the frequency of these traps.

The hypervisor extension has been designed to be efficiently
emulable on platforms that do not implement the extension, by running
the hypervisor in S-mode and trapping into M-mode for hypervisor CSR accesses
and to maintain shadow page tables.  The majority of CSR accesses for
type-2 hypervisors are valid S-mode accesses so need not be trapped.
Hypervisors can support nested virtualization analogously.
\end{commentary}

\section{Privilege Modes}

The current {\em virtualization mode}, denoted V, indicates whether the hart
is currently executing in a guest.
When V=1, the hart is either in virtual S-mode (VS-mode), or in virtual U-mode
(VU-mode) under a guest OS running in VS-mode.
When V=0, the hart is either in M-mode, in HS-mode, or in U-mode under an OS
running in HS-mode.
The virtualization mode also indicates whether two-level address translation
is active (V=1) or inactive (V=0).  Table~\ref{h-operating-modes} lists the
possible operating modes of a RISC-V hart with the hypervisor extension.

\begin{table*}[h!]
\begin{center}
\begin{tabular}{|c|c||l|l|l|}
  \hline
   Virtualization & Privilege & \multirow{2}{*}{Abbreviation} & \multirow{2}{*}{Name} & Two-Level \\
   Mode (V)       & Encoding  &                               &                       & Translation \\ \hline
   0              & 0         & U-mode  & User mode & Off \\
   0              & 1         & HS-mode & Hypervisor-extended supervisor mode & Off \\
   0              & 3         & M-mode  & Machine mode & Off \\
  \hline
   1              & 0         & VU-mode & Virtual user mode & On \\
   1              & 1         & VS-mode & Virtual supervisor mode & On \\
  \hline
 \end{tabular}
\end{center}
\caption{Operating modes with the hypervisor extension.}
\label{h-operating-modes}
\end{table*}

\section{Hypervisor CSRs}

An OS or hypervisor running in HS-mode uses the supervisor CSRs to interact with the exception,
interrupt, and address-translation subsystems.
Additional CSRs are provided to HS-mode, but not to VS-mode, to manage
two-level address translation and to control the behavior of a VS-mode guest:
{\tt hstatus}, {\tt hedeleg}, {\tt hideleg}, and {\tt hgatp}.

Additionally, several {\em background} supervisor CSRs are copies of one of
the existing {\em foreground} supervisor CSRs.  For example, the {\tt
bsstatus} CSR is the background copy of the foreground {\tt sstatus} CSR.
When transitioning between virtualization modes (V=0 to V=1, or vice-versa),
the implementation swaps the background supervisor CSRs with their foreground
counterparts.  When V=0, the background supervisor CSRs contain VS-mode's
version of those CSRs, and the foreground supervisor CSRs contain
HS-mode's version.  When V=1, the background supervisor CSRs contain HS-mode's
version, and the foreground supervisor CSRs contain VS-mode's
version.  The background registers are accessible to HS-mode, but not to
VS-mode.

\begin{commentary}
The swapping of foreground and background supervisor registers can be
implemented either by physically copying bits or by dynamically changing the
CSR addresses of hardware registers.
The CSR addresses of the background supervisor registers have been aligned with
their foreground counterparts to minimize the cost of swapping registers simply
by changing their addresses.
\end{commentary}

In this section, we use the term {\em HSXLEN} to refer to the effective XLEN
when executing in HS-mode.

\subsection{Hypervisor Status Register ({\tt hstatus})}

The {\tt hstatus} register is an HSXLEN-bit read/write register
formatted as shown in Figure~\ref{hstatusreg}.  The {\tt hstatus}
register provides facilities analogous to the {\tt mstatus} register
that track and control the exception behavior of a VS-mode guest.

\begin{figure*}[h!]
{\footnotesize
\begin{center}
\setlength{\tabcolsep}{4pt}
\begin{tabular}{ScccRccccFc}
\\
\instbitrange{HSXLEN-1}{23} &
\instbit{22} &
\instbit{21} &
\instbit{20} &
\instbitrange{19}{11} &
\instbit{10} &
\instbit{9} &
\instbit{8} &
\instbit{7} &
\instbitrange{6}{1} &
\instbit{0} \\
\hline
\multicolumn{1}{|c|}{\wpri} &
\multicolumn{1}{c|}{VTSR} &
\multicolumn{1}{c|}{VTW} &
\multicolumn{1}{c|}{VTVM} &
\multicolumn{1}{c|}{\wpri} &
\multicolumn{1}{c|}{SPV} &
\multicolumn{1}{c|}{STL} &
\multicolumn{1}{c|}{SP2P} &
\multicolumn{1}{c|}{SP2V} &
\multicolumn{1}{c|}{\wpri} &
\multicolumn{1}{c|}{SPRV} \\
\hline
HSXLEN-23 & 1 & 1 & 1 & 9 & 1 & 1 & 1 & 1 & 6 & 1 \\
\end{tabular}
\end{center}
}
\vspace{-0.1in}
\caption{Hypervisor-mode status register ({\tt hstatus}).}
\label{hstatusreg}
\end{figure*}

The {\tt hstatus} fields VTSR, VTW, and VTVM are defined analogously to the
{\tt mstatus} fields TSR, TW, and TVM, but affect the trapping behavior of the
SRET, WFI, and virtual-memory management instructions only when V=1.

The SPV bit (Supervisor Previous Virtualization Mode) is written by the implementation
whenever a trap is taken into HS-mode.  Just as the SPP bit in {\tt sstatus} is set to the privilege
mode at the time of the trap, the SPV bit in {\tt hstatus} is set to the value of the virtualization
mode V at the time of the trap.  When an SRET instruction is executed when V=0,
V is set to SPV.

When a trap is taken into HS-mode, bits SP2V and SP2P are set to the values
that SPV and the HS-level SPP had before the trap.
(Before the trap, the HS-level SPP is {\tt sstatus}.SPP if V=0, or
{\tt bsstatus}.SPP if V=1.)
When an SRET instruction is executed when V=0, the reverse assignments occur:
after SPV and {\tt sstatus}.SPP have supplied the new virtualization and
privilege modes, they are written with SP2V and SP2P, respectively.

The STL bit (Supervisor Translation Level), which indicates which address-translation level
caused an access-fault or page-fault exception, is also written by the implementation whenever a trap
is taken into HS-mode.
On an access or page fault due to guest physical address translation, STL is
set to 1.
For any other trap into HS-mode, STL is set to 0.

The SPRV bit modifies the privilege with which loads and stores execute when
not in M-mode.
When SPRV=0, translation and protection behave as normal.  When SPRV=1,
load and store memory addresses are translated and protected as though
the current virtualization mode were set to {\tt hstatus}.SPV and the current
privilege mode were set to the HS-level SPP ({\tt sstatus}.SPP when V=0, or
{\tt bsstatus}.SPP when V=1).
Table~\ref{h-sprv} enumerates the cases.

\begin{table*}[h!]
\begin{center}
\begin{tabular}{|c|c|c||p{4.7in}|}
  \hline
   SPRV & SPV & SPP & Effect \\ \hline \hline
   0    & --  & --  & Normal access; current privilege and virtualization modes apply. \\ \hline
   1    & 0   & 0   & U-level access with HS-level translation and protection only. \\ \hline
   1    & 0   & 1   & HS-level access with HS-level translation and protection only. \\ \hline
   1    & 1   & 0   & VU-level access with two-level translation and protection. The HS-level MXR bit makes any executable page readable.  The VS-level MXR makes readable those pages marked executable at the VS translation level only if readable at the guest-physical translation level. \\ \hline
   1    & 1   & 1   & VS-level access with two-level translation and protection. The HS-level MXR bit makes any executable page readable.  The VS-level MXR makes readable those pages marked executable at the VS translation level only if readable at the guest-physical translation level.  The VS-level SUM bit applies instead of HS-level SUM. \\ \hline
 \end{tabular}
\end{center}
\caption{Effect on load and store translation and protection under SPRV.}
\label{h-sprv}
\end{table*}

\begin{commentary}
For simplicity, SPRV is in effect even when in U-mode, VS-mode, or VU-mode, but
in normal use will only be enabled for short sequences in HS-mode.
\end{commentary}

\subsection{Hypervisor Trap Delegation Registers ({\tt hedeleg} and {\tt hideleg})}

By default, all traps at any privilege level are handled in M-mode, though
M-mode usually uses the {\tt medeleg} and {\tt mideleg} CSRs to delegate
some traps to HS-mode.  The {\tt hedeleg} and {\tt hideleg} CSRs allow these
traps to be further delegated to a VS-mode guest; their layout is the same
as {\tt medeleg} and {\tt mideleg}.

\begin{figure}[h!]
{\footnotesize
\begin{center}
\begin{tabular}{@{}U}
\instbitrange{HSXLEN-1}{0} \\
\hline
\multicolumn{1}{|c|}{Synchronous Exceptions (\warl)} \\
\hline
HSXLEN \\
\end{tabular}
\end{center}
}
\vspace{-0.1in}
\caption{Hypervisor Exception Delegation Register {\tt hedeleg}.}
\label{hedelegreg}
\end{figure}

\begin{figure}[h!]
{\footnotesize
\begin{center}
\begin{tabular}{@{}U}
\instbitrange{HSXLEN-1}{0} \\
\hline
\multicolumn{1}{|c|}{Interrupts (\warl)} \\
\hline
HSXLEN \\
\end{tabular}
\end{center}
}
\vspace{-0.1in}
\caption{Hypervisor Interrupt Delegation Register {\tt hideleg}.}
\label{hidelegreg}
\end{figure}

The {\tt hedeleg} and {\tt hideleg} registers are only active when V=1.  When
V=1, a trap that has been delegated to HS-mode (using {\tt medeleg} or {\tt
mideleg}) is further delegated to VS-mode if the corresponding {\tt hedeleg} or
{\tt hideleg} bit is set.  If the N extension for user-mode interrupts
is implemented, the VS-mode guest may further delegate the interrupt
to VU-mode by setting the corresponding bit in {\tt sedeleg} or {\tt sideleg}.

When V=0 and the N extension for user-mode interrupts is implemented, a trap
that has been delegated to HS-mode can be further delegated to U-mode by
setting the corresponding bit in {\tt sedeleg} or {\tt sideleg}.

When an access-fault or page-fault exception is caused by guest physical
address translation, the trap is not delegated beyond HS-mode, regardless of
the setting of {\tt hedeleg}.

\subsection{Hypervisor Guest Address Translation and Protection Register ({\tt hgatp})}
\label{sec:hgatp}

The {\tt hgatp} register is an HSXLEN-bit read/write register, formatted as
shown in Figure~\ref{rv32hgatp} for HSXLEN=32 and Figure~\ref{rv64hgatp} for
HSXLEN=64, which controls guest physical address translation and protection.
Similar to CSR {\tt satp}, this register holds the physical page number (PPN)
of the guest-physical root page table; a virtual machine identifier (VMID),
which facilitates address-translation fences on a per-virtual-machine basis;
and the MODE field, which selects the address-translation scheme for guest
physical addresses.
When {\tt mstatus}.TVM=1, attempts to read or write {\tt hgatp} while executing
in HS-mode will raise an illegal instruction exception.

\begin{figure}[h!]
{\footnotesize
\begin{center}
\begin{tabular}{cY@{}E@{}K}
\instbit{31} &
\instbitrange{30}{29} &
\instbitrange{28}{22} &
\instbitrange{21}{0} \\
\hline
\multicolumn{1}{|c|}{MODE} &
\multicolumn{1}{|c|}{0 (\warl)} &
\multicolumn{1}{|c|}{VMID (\warl)} &
\multicolumn{1}{|c|}{PPN  (\warl)} \\
\hline
1 & 2 & 7 & 22 \\
\end{tabular}
\end{center}
}
\vspace{-0.1in}
\caption{RV32 Hypervisor guest address translation and protection register
{\tt hgatp}.}
\label{rv32hgatp}
\end{figure}

\begin{figure}[h!]
{\footnotesize
\begin{center}
\begin{tabular}{@{}S@{}Y@{}E@{}K}
\instbitrange{63}{60} &
\instbitrange{59}{58} &
\instbitrange{57}{44} &
\instbitrange{43}{0} \\
\hline
\multicolumn{1}{|c|}{MODE (\warl)} &
\multicolumn{1}{|c|}{0 (\warl)} &
\multicolumn{1}{|c|}{VMID (\warl)} &
\multicolumn{1}{|c|}{PPN  (\warl)} \\
\hline
4 & 2 & 14 & 44 \\
\end{tabular}
\end{center}
}
\vspace{-0.1in}
\caption{RV64 Hypervisor guest address translation and protection register
{\tt hgatp}, for MODE values Bare, Sv39x4, and Sv48x4.}
\label{rv64hgatp}
\end{figure}

Table~\ref{tab:hgatp-mode} shows the encodings of the MODE field for RV32 and
RV64.
When MODE=Bare, guest physical addresses are equal to supervisor physical
addresses, and there is no further memory protection for a guest virtual
machine beyond the physical memory protection scheme described in
Section~\ref{sec:pmp}.
In this case, the remaining fields in {\tt hgatp} have no effect.

For RV32, the only other valid setting for MODE is Sv32x4, which is a
modification of the usual Sv32 paged virtual-memory scheme, extended to support
34-bit guest physical addresses.
For RV64, modes Sv39x4 and Sv48x4 are defined as modifications of the Sv39 and
Sv48 paged virtual-memory schemes.
All these paged virtual-memory schemes are described in
Section~\ref{sec:guest-addr-translation}.
An additional RV64 scheme, Sv57x4, may be defined in a later version of this
specification.

The remaining MODE settings for RV64 are reserved for future use and may define
different interpretations of the other fields in {\tt hgatp}.

RV64 implementations are not required to support all defined RV64 MODE
settings.
(However, a write to {\tt hgatp} with an unsupported MODE value is not ignored
as it is for {\tt satp}.)

\begin{table}[h]
\begin{center}
\begin{tabular}{|c|c|l|}
\hline
\multicolumn{3}{|c|}{RV32} \\
\hline
Value  & Name & Description \\
\hline
0      & Bare   & No translation or protection. \\
1      & Sv32x4 & Page-based 34-bit virtual addressing (2-bit extension of Sv32). \\
\hline \hline
\multicolumn{3}{|c|}{RV64} \\
\hline
Value  & Name & Description \\
\hline
0      & Bare   & No translation or protection. \\
1--7   & ---    & {\em Reserved} \\
8      & Sv39x4 & Page-based 41-bit virtual addressing (2-bit extension of Sv39). \\
9      & Sv48x4 & Page-based 50-bit virtual addressing (2-bit extension of Sv48). \\
10     & {\em Sv57x4} & {\em Reserved for page-based 59-bit virtual addressing.} \\
11--15 & ---    & {\em Reserved} \\
\hline
\end{tabular}
\end{center}
\caption{Encoding of {\tt hgatp} MODE field.}
\label{tab:hgatp-mode}
\end{table}

As explained in Section~\ref{sec:guest-addr-translation}, for the paged
virtual-memory schemes (Sv32x4, Sv39x4, and Sv48x4), the root page table is
16~KiB and must be aligned to a 16-KiB boundary.
In these modes, the lowest two bits of the physical page number (PPN) in
{\tt hgatp} are ignored.
An implementation that supports only the defined paged virtual-memory schemes
and/or Bare may hardwire PPN[1:0] to zero.

The number of supervisor physical address bits is implementation-defined; any
unimplemented address bits are hardwired to zero in {\tt hgatp}.PPN.
The number of VMID bits is also implementation-defined and may be zero.
The number of implemented VMID bits, termed {\mbox {\em VMIDLEN}}, may be
determined by writing one to every bit position in the VMID field, then reading
back the value in {\tt hgatp} to see which bit positions in the VMID field hold
a one.
The least-significant bits of VMID are implemented first:
that is, if VMIDLEN~$>$~0, VMID[VMIDLEN-1:0] is writable.
The maximal value of VMIDLEN, termed VMIDMAX, is 7 for Sv32x4 or 14 for Sv39x4
and Sv48x4.

Note that writing {\tt hgatp} does not imply any ordering constraints between
page-table updates and subsequent guest physical address translations.
If the new virtual machine's guest physical page tables have been modified, it
may be necessary to execute an HFENCE.GVMA instruction
(see Section~\ref{sec:hfence.vma}) before or after writing {\tt hgatp}.

\subsection{Background Supervisor Status Register ({\tt bsstatus})}

The {\tt bsstatus} register is an HSXLEN-bit read/write register formatted as
shown in Figure~\ref{bsstatusreg}.  When V=0, the {\tt bsstatus} register
holds VS-mode's version of several fields of the {\tt sstatus}
register: UXL, MXR, SUM, FS, SPP, SPIE, and SIE.  When V=1, {\tt bsstatus}
holds HS-mode's version of these fields.  When transitioning between
virtualization modes (V=0 to V=1, or vice-versa), the implementation swaps
these fields in {\tt bsstatus} with their counterparts in {\tt sstatus}.  The
other fields in {\tt sstatus} are unchanged.

When V=1, both {\tt bsstatus}.FS and {\tt sstatus}.FS are in effect.  Attempts
to execute a floating-point instruction when either field is 0 (Off) raise an
illegal-instruction exception.  Modifying the floating-point state when V=1
causes both fields to be set to 3 (Dirty).

When V=0, {\tt bsstatus} does not directly affect the behavior of the machine,
unless the MPRV feature in the {\tt mstatus} register or the SPRV feature
in the {\tt hstatus} register is used to execute a load or store
{\em as though} V=1.

\begin{figure*}[h!]
{\footnotesize
\begin{center}
\setlength{\tabcolsep}{4pt}
\begin{tabular}{McEccc}
\\
\instbitrange{HSXLEN-1}{34} &
\instbitrange{33}{32} &
\instbitrange{31}{20} &
\instbit{19} &
\instbit{18} &
 \\
\hline
\multicolumn{1}{|c|}{\wpri} &
\multicolumn{1}{c|}{UXL} &
\multicolumn{1}{c|}{\wpri} &
\multicolumn{1}{c|}{MXR} &
\multicolumn{1}{c|}{SUM} &
 \\
\hline
HSXLEN-34 & 2 & 12 & 1 & 1 & \\
\end{tabular}
\begin{tabular}{cFFYcWcFcc}
\\
&
\instbitrange{17}{15} &
\instbitrange{14}{13} &
\instbitrange{12}{9} &
\instbit{8} &
\instbitrange{7}{6} &
\instbit{5} &
\instbitrange{4}{2} &
\instbit{1} &
\instbit{0} \\
\hline
 &
\multicolumn{1}{c|}{\wpri} &
\multicolumn{1}{c|}{FS[1:0]} &
\multicolumn{1}{c|}{\wpri} &
\multicolumn{1}{c|}{SPP} &
\multicolumn{1}{c|}{\wpri} &
\multicolumn{1}{c|}{SPIE} &
\multicolumn{1}{c|}{\wpri} &
\multicolumn{1}{c|}{SIE} &
\multicolumn{1}{c|}{\wpri} \\
\hline
 & 3 & 2 & 4 & 1 & 2 & 1 & 3 & 1 & 1 \\
\end{tabular}
\end{center}
}
\vspace{-0.1in}
\caption{Background supervisor status register ({\tt bsstatus}) for RV64 and RV128.}
\label{bsstatusreg}
\end{figure*}

\subsection{Background Supervisor Interrupt Registers ({\tt bsip} and {\tt bsie})}

The {\tt bsip} register is an HSXLEN-bit read/write register formatted as shown
in Figure~\ref{bsipreg}.  When V=0, the {\tt bsip} register holds VS-mode's
version of the {\tt sip} register.  When V=1, {\tt bsip} holds
HS-mode's version of the {\tt sip} register.  When transitioning between
virtualization modes (V=0 to V=1, or vice-versa), the implementation swaps the
defined fields of {\tt bsip} with their counterparts in {\tt sip}.  The
other fields in {\tt sip} are unchanged.

\note{AW: Need to describe how {\tt bsip}.SEIP interacts with PLIC.  I think {\tt bsip}.SEIP should purely be a read-write storage bit to emulate the PLIC for VS-mode; the PLIC should not be wired into {\tt bsip}.SEIP.}

\begin{figure*}[h!]
{\footnotesize
\begin{center}
\setlength{\tabcolsep}{4pt}
\begin{tabular}{TcFcFcc}
\instbitrange{HSXLEN-1}{10} &
\instbit{9} &
\instbitrange{8}{6} &
\instbit{5} &
\instbitrange{4}{2} &
\instbit{1} &
\instbit{0} \\
\hline
\multicolumn{1}{|c|}{\wpri} &
\multicolumn{1}{c|}{SEIP} &
\multicolumn{1}{c|}{\wpri} &
\multicolumn{1}{c|}{STIP} &
\multicolumn{1}{c|}{\wpri} &
\multicolumn{1}{c|}{SSIP} &
\multicolumn{1}{c|}{\wpri} \\
\hline
HSXLEN-10 & 1 & 3 & 1 & 3 & 1 & 1 \\
\end{tabular}
\end{center}
}
\vspace{-0.1in}
\caption{Background supervisor interrupt-pending register ({\tt bsip}).}
\label{bsipreg}
\end{figure*}

The {\tt bsie} register is an HSXLEN-bit read/write register formatted as shown
in Figure~\ref{bsiereg}.  When V=0, the {\tt bsie} register holds VS-mode's
version of the {\tt sie} register.  When V=1, {\tt bsie} holds
HS-mode's version of the {\tt sie} register. When transitioning between
virtualization modes (V=0 to V=1, or vice-versa), the implementation swaps the
defined fields of {\tt bsie} with their counterparts in {\tt sie}.  The
other fields in {\tt sie} are unchanged.

\begin{figure*}[h!]
{\footnotesize
\begin{center}
\setlength{\tabcolsep}{4pt}
\begin{tabular}{TcFcFcc}
\instbitrange{HSXLEN-1}{10} &
\instbit{9} &
\instbitrange{8}{6} &
\instbit{5} &
\instbitrange{4}{2} &
\instbit{1} &
\instbit{0} \\
\hline
\multicolumn{1}{|c|}{\wpri} &
\multicolumn{1}{c|}{SEIE} &
\multicolumn{1}{c|}{\wpri} &
\multicolumn{1}{c|}{STIE} &
\multicolumn{1}{c|}{\wpri} &
\multicolumn{1}{c|}{SSIE} &
\multicolumn{1}{c|}{\wpri} \\
\hline
HSXLEN-10 & 1 & 3 & 1 & 3 & 1 & 1 \\
\end{tabular}
\end{center}
}
\vspace{-0.1in}
\caption{Background supervisor interrupt-enable register ({\tt bsie}).}
\label{bsiereg}
\end{figure*}

When V=0, {\tt bsip} and {\tt bsie} do not affect the behavior of the machine.
When V=1, they hold the active interrupt-pending and interrupt-enable bits,
respectively, for HS-mode; if any bit position holds a 1 in both registers, an
interrupt will be taken.

\begin{commentary}
The {\tt bsip} and {\tt bsie} CSRs do not hold copies of the user-mode
interrupt fields.  The expectation is that the context-switch code
will swap the {\tt uip} and {\tt uie} CSRs
along with the other user-mode interrupt
registers ({\tt ustatus}, {\tt utvec}, etc.) if that feature is enabled.
\end{commentary}

\subsection{Background Supervisor Trap Vector Base Address Register ({\tt bstvec})}

The {\tt bstvec} register is an HSXLEN-bit read/write register formatted as shown
in Figure~\ref{bstvecreg}.  When V=0, the {\tt bstvec} register holds VS-mode's
version of the {\tt stvec} register.  When V=1, {\tt bstvec}
holds HS-mode's version of the {\tt stvec} register.  When transitioning between
virtualization modes (V=0 to V=1, or vice-versa), the implementation swaps the
contents of {\tt bstvec} and {\tt stvec}.

When V=0, {\tt bstvec} does not directly affect the behavior of the machine.  When V=1,
it controls the value to which the {\tt pc} will be set upon a trap into
HS-mode.

\begin{figure*}[h!]
{\footnotesize
\begin{center}
\begin{tabular}{J@{}R}
\instbitrange{HSXLEN-1}{2} &
\instbitrange{1}{0} \\
\hline
\multicolumn{1}{|c|}{BASE[HSXLEN-1:2] (\warl)} &
\multicolumn{1}{c|}{MODE (\warl)} \\
\hline
HSXLEN-2 & 2 \\
\end{tabular}
\end{center}
}
\vspace{-0.1in}
\caption{Background supervisor trap vector base address register ({\tt bstvec}).}
\label{bstvecreg}
\end{figure*}

\subsection{Background Supervisor Scratch Register ({\tt bsscratch})}

The {\tt bsscratch} register is an HSXLEN-bit read/write register formatted as shown
in Figure~\ref{bsscratchreg}.  When V=0, the {\tt bsscratch} register holds
VS-mode's version of the {\tt sscratch} register.  When V=1, {\tt bsscratch}
holds HS-mode's version of the {\tt sscratch} register.  When transitioning between
virtualization modes (V=0 to V=1, or vice-versa), the implementation swaps the
contents of {\tt bsscratch} and {\tt sscratch}.

Typically, {\tt bsscratch} is used to hold a pointer to the hart-local
hypervisor context (when V=1) or supervisor context (when V=0).  The
contents of {\tt bsscratch} do not directly affect the behavior of
the machine.

\begin{figure*}[h!]
{\footnotesize
\begin{center}
\begin{tabular}{@{}J}
\instbitrange{HSXLEN-1}{0} \\
\hline
\multicolumn{1}{|c|}{\tt bsscratch} \\
\hline
HSXLEN \\
\end{tabular}
\end{center}
}
\vspace{-0.1in}
\caption{Background supervisor scratch register ({\tt bsscratch}).}
\label{bsscratchreg}
\end{figure*}

\subsection{Background Supervisor Exception Program Counter ({\tt bsepc})}

The {\tt bsepc} register is an HSXLEN-bit read/write register formatted as shown
in Figure~\ref{bsepcreg}.  When V=0, the {\tt bsepc} register holds
VS-mode's version of the {\tt sepc} register.  When V=1, {\tt bsepc}
holds HS-mode's version of the {\tt sepc} register.  When transitioning between
virtualization modes (V=0 to V=1, or vice-versa), the implementation swaps the
contents of {\tt bsepc} and {\tt sepc}.

The contents of {\tt bsepc} do not directly affect the behavior of
the machine.

{\tt bsepc} is a \warl\ register that must be able to hold the same set of
values that {\tt sepc} can hold.

\begin{figure*}[h!]
{\footnotesize
\begin{center}
\begin{tabular}{@{}J}
\instbitrange{HSXLEN-1}{0} \\
\hline
\multicolumn{1}{|c|}{\tt bsepc} \\
\hline
HSXLEN \\
\end{tabular}
\end{center}
}
\vspace{-0.1in}
\caption{Background supervisor exception program counter ({\tt bsepc}).}
\label{bsepcreg}
\end{figure*}

\subsection{Background Supervisor Cause Register ({\tt bscause})}

The {\tt bscause} register is an HSXLEN-bit read/write register formatted as shown
in Figure~\ref{bscausereg}.  When V=0, the {\tt bscause} register holds
VS-mode's version of the {\tt scause} register.  When V=1, {\tt bscause}
holds HS-mode's version of the {\tt scause} register.  When transitioning between
virtualization modes (V=0 to V=1, or vice-versa), the implementation swaps the
contents of {\tt bscause} and {\tt scause}.

The contents of {\tt bscause} do not directly affect the behavior of
the machine.

{\tt bscause} is a \wlrl\ register that must be able to hold the same set of
values that {\tt scause} can hold.

\begin{figure*}[h!]
{\footnotesize
\begin{center}
\begin{tabular}{c@{}U}
\instbit{HSXLEN-1} &
\instbitrange{HSXLEN-2}{0} \\
\hline
\multicolumn{1}{|c|}{Interrupt} &
\multicolumn{1}{c|}{Exception Code (\wlrl)} \\
\hline
1 & HSXLEN-1 \\
\end{tabular}
\end{center}
}
\vspace{-0.1in}
\caption{Background supervisor cause register ({\tt bscause}).}
\label{bscausereg}
\end{figure*}

\subsection{Background Supervisor Trap Value Register ({\tt bstval})}

The {\tt bstval} register is an HSXLEN-bit read/write register formatted as shown
in Figure~\ref{bstvalreg}.  When V=0, the {\tt bstval} register holds
VS-mode's version of the {\tt stval} register.  When V=1, {\tt bstval}
holds HS-mode's version of the {\tt stval} register.  When transitioning between
virtualization modes (V=0 to V=1, or vice-versa), the implementation swaps the
contents of {\tt bstval} and {\tt stval}.

The contents of {\tt bstval} do not directly affect the behavior of
the machine.

{\tt bstval} is a \warl\ register that must be able to hold the same set of
values that {\tt stval} can hold.

\begin{figure*}[h!]
{\footnotesize
\begin{center}
\begin{tabular}{@{}J}
\instbitrange{HSXLEN-1}{0} \\
\hline
\multicolumn{1}{|c|}{\tt bstval} \\
\hline
HSXLEN \\
\end{tabular}
\end{center}
}
\vspace{-0.1in}
\caption{Background supervisor trap value register ({\tt bstval}).}
\label{bstvalreg}
\end{figure*}

\subsection{Background Supervisor Address Translation and Protection Register ({\tt bsatp})}

The {\tt bsatp} register is an HSXLEN-bit read/write register formatted as shown
in Figure~\ref{rv32bsatpreg} for RV32 and Figure~\ref{rv64bsatpreg}  for RV64.
When V=0, the {\tt bsatp} register holds VS-mode's version of the
{\tt satp} register.  When V=1, {\tt bsatp} holds HS-mode's version of the
{\tt satp} register.  When transitioning between virtualization modes (V=0 to
V=1, or vice-versa), the implementation swaps the contents of {\tt bsatp} and
{\tt satp}.

{\tt bsatp} does not directly affect the behavior of the machine, unless the
MPRV feature in the {\tt mstatus} register or the SPRV feature in the
{\tt hstatus} register is used to execute a load or store {\em as though} V is
the opposite of its actual setting.
The interpretation of the MODE, ASID, and PPN
fields is the same as for {\tt satp}.

\begin{figure}[h!]
{\footnotesize
\begin{center}
\begin{tabular}{c@{}E@{}K}
\instbit{31} &
\instbitrange{30}{22} &
\instbitrange{21}{0} \\
\hline
\multicolumn{1}{|c|}{MODE (\warl)} &
\multicolumn{1}{|c|}{ASID (\warl)} &
\multicolumn{1}{|c|}{PPN  (\warl)} \\
\hline
1 & 9 & 22 \\
\end{tabular}
\end{center}
}
\vspace{-0.1in}
\caption{RV32 background supervisor address translation and protection register {\tt bsatp}.}
\label{rv32bsatpreg}
\end{figure}

\begin{figure*}[h!]
{\footnotesize
\begin{center}
\begin{tabular}{@{}S@{}T@{}U}
\instbitrange{63}{60} &
\instbitrange{59}{44} &
\instbitrange{43}{0} \\
\hline
\multicolumn{1}{|c|}{MODE (\warl)} &
\multicolumn{1}{|c|}{ASID (\warl)} &
\multicolumn{1}{|c|}{PPN  (\warl)} \\
\hline
4 & 16 & 44 \\
\end{tabular}
\end{center}
}
\vspace{-0.1in}
\caption{RV64 background supervisor address translation and protection register {\tt bsatp}, for MODE
values Bare, Sv39, and Sv48.}
\label{rv64bsatpreg}
\end{figure*}

\section{Hypervisor Instructions}

The hypervisor extension adds two new privileged fence instructions.

\subsection{Hypervisor Memory-Management Fence Instructions}
\label{sec:hfence.vma}

\vspace{-0.2in}
\begin{center}
\begin{tabular}{O@{}R@{}R@{}F@{}R@{}S}
\\
\instbitrange{31}{25} &
\instbitrange{24}{20} &
\instbitrange{19}{15} &
\instbitrange{14}{12} &
\instbitrange{11}{7} &
\instbitrange{6}{0} \\
\hline
\multicolumn{1}{|c|}{funct7} &
\multicolumn{1}{c|}{rs2} &
\multicolumn{1}{c|}{rs1} &
\multicolumn{1}{c|}{funct3} &
\multicolumn{1}{c|}{rd} &
\multicolumn{1}{c|}{opcode} \\
\hline
7 & 5 & 5 & 3 & 5 & 7 \\
HFENCE.GVMA & vmid & gaddr & PRIV & 0 & SYSTEM \\
HFENCE.BVMA & asid & vaddr & PRIV & 0 & SYSTEM \\
\end{tabular}
\end{center}

The hypervisor memory-management fence instructions, HFENCE.GVMA and
HFENCE.BVMA, are valid only in HS-mode when {\tt mstatus}.TVM=0, or in M-mode
(irrespective of {\tt mstatus}.TVM).
These instructions perform a function similar to SFENCE.VMA
(Section~\ref{sec:sfence.vma}), except applying to the guest-physical
memory-management data structures controlled by CSR {\tt hgatp} (HFENCE.GVMA)
or the VS-level memory-management data structures controlled by CSR {\tt bsatp}
(HFENCE.BVMA).
Instruction SFENCE.VMA applies only to the memory-management data structures
controlled by the foreground {\tt satp}.

If an HFENCE.BVMA instruction executes without trapping, its effect is much the
same as temporarily entering VS-mode (with the usual swapping of foreground and
background supervisor registers) and executing SFENCE.VMA.
Executing an HFENCE.BVMA guarantees that any stores in the instruction stream
prior to the HFENCE.BVMA are ordered before implicit references to VS-level
memory-management data structures when those implicit references
\begin{compactitem}
\item
are subsequent to the HFENCE.BVMA, and
\item
occur when {\tt hgatp}.VMID has the same setting as it did when HFENCE.BVMA
executed.
\end{compactitem}
Implicit references need not be ordered when {\tt hgatp}.VMID is different than
at the time HFENCE.BVMA executed.
If operand {\em rs1}$\neq${\tt x0}, it specifies a single guest virtual
address, and if operand {\em rs2}$\neq${\tt x0}, it specifies a single guest
address-space identifier
(ASID).

\begin{commentary}
An HFENCE.BVMA instruction applies only to a single virtual machine, identified
by the setting of {\tt hgatp}.VMID when HFENCE.BVMA executes.
\end{commentary}

When {\em rs2}$\neq${\tt x0}, bits XLEN-1:ASIDMAX of the value held in {\em
rs2} are reserved for future use and should be zeroed by software and ignored
by current implementations.
Furthermore, if ASIDLEN~$<$~ASIDMAX, the implementation shall ignore bits
ASIDMAX-1:ASIDLEN of the value held in {\em rs2}.

\begin{commentary}
Simpler implementations of HFENCE.BVMA can ignore the guest virtual address in
{\em rs1} and the guest ASID value in {\em rs2}, as well as {\tt hgatp}.VMID,
and always perform a global fence for the VS-level memory management of all
virtual machines, or even a global fence for all memory-management data
structures.
\end{commentary}

Executing an HFENCE.GVMA instruction guarantees that any stores in the
instruction stream prior to the HFENCE.GVMA are ordered before all implicit
references to guest-physical memory-management data structures subsequent to
the HFENCE.GVMA.
If operand {\em rs1}$\neq${\tt x0}, it specifies a single guest physical
address, shifted right by 2~bits, and if operand {\em rs2}$\neq${\tt x0}, it
specifies a single virtual machine identifier (VMID).

\begin{commentary}
For HFENCE.GVMA, a guest physical address specified in {\em rs1} is shifted
right by 2~bits to accommodate addresses wider than the current XLEN.
For RV32, the hypervisor extension permits guest physical addresses as wide as
34 bits, and {\em rs1} specifies bits 33:2 of such an address.
This shift-by-2 encoding of guest physical addresses matches the encoding of
physical addresses in PMP address registers (Section~\ref{sec:pmp}) and in page
table entries (Sections \ref{sec:sv32}, \ref{sec:sv39}, and~\ref{sec:sv48}).
\end{commentary}

When {\em rs2}$\neq${\tt x0}, bits XLEN-1:VMIDMAX of the value held in {\em
rs2} are reserved for future use and should be zeroed by software and ignored
by current implementations.
Furthermore, if VMIDLEN~$<$~VMIDMAX, the implementation shall ignore bits
VMIDMAX-1:VMIDLEN of the value held in {\em rs2}.

\begin{commentary}
Simpler implementations of HFENCE.GVMA can ignore the guest physical address in
{\em rs1} and the VMID value in {\em rs2} and always perform a global fence for
the guest-physical memory management of all virtual machines, or even a global
fence for all memory-management data structures.
\end{commentary}

\section{Machine-Level CSRs}

The hypervisor extension augments the {\tt mstatus} CSR.

\subsection{Machine Status Register ({\tt mstatus})}

The hypervisor extension adds two fields to the machine-mode {\tt mstatus} CSR,
MPV and MTL,
and modifies the behavior of several existing fields.
Figure~\ref{hypervisor-mstatus} shows the {\tt mstatus} register when the
hypervisor extension is provided.

\begin{figure*}[h!]
{\footnotesize
\begin{center}
\setlength{\tabcolsep}{4pt}
\begin{tabular}{cRccYccccccc}
\\
\instbit{MXLEN-1} &
\instbitrange{MXLEN-2}{36} &
\instbitrange{35}{34} &
\instbitrange{33}{32} &
\instbitrange{31}{23} &
\instbit{22} &
\instbit{21} &
\instbit{20} &
\instbit{19} &
\instbit{18} &
\instbit{17} &
 \\
\hline
\multicolumn{1}{|c|}{SD} &
\multicolumn{1}{c|}{\wpri} &
\multicolumn{1}{c|}{SXL[1:0]} &
\multicolumn{1}{c|}{UXL[1:0]} &
\multicolumn{1}{c|}{\wpri} &
\multicolumn{1}{c|}{TSR} &
\multicolumn{1}{c|}{TW} &
\multicolumn{1}{c|}{TVM} &
\multicolumn{1}{c|}{MXR} &
\multicolumn{1}{c|}{SUM} &
\multicolumn{1}{c|}{MPRV} &
 \\
\hline
1 & MXLEN-37 & 2 & 2 & 9 & 1 & 1 & 1 & 1 & 1 & 1 & \\
\end{tabular}
\begin{tabular}{ccccccccccccccc}
\\
&
\instbitrange{16}{15} &
\instbitrange{14}{13} &
\instbitrange{12}{11} &
\instbit{10} &
\instbit{9} &
\instbit{8} &
\instbit{7} &
\instbit{6} &
\instbit{5} &
\instbit{4} &
\instbit{3} &
\instbit{2} &
\instbit{1} &
\instbit{0} \\
\hline
 &
\multicolumn{1}{|c|}{XS[1:0]} &
\multicolumn{1}{c|}{FS[1:0]} &
\multicolumn{1}{c|}{MPP[1:0]} &
\multicolumn{1}{c|}{MPV} &
\multicolumn{1}{c|}{MTL} &
\multicolumn{1}{c|}{SPP} &
\multicolumn{1}{c|}{MPIE} &
\multicolumn{1}{c|}{\wpri} &
\multicolumn{1}{c|}{SPIE} &
\multicolumn{1}{c|}{UPIE} &
\multicolumn{1}{c|}{MIE} &
\multicolumn{1}{c|}{\wpri} &
\multicolumn{1}{c|}{SIE} &
\multicolumn{1}{c|}{UIE} \\
\hline
 & 2 & 2 & 2 & 1 & 1 & 1 & 1 & 1 & 1 & 1 & 1 & 1 & 1 & 1 \\
\end{tabular}
\end{center}
}
\vspace{-0.1in}
\caption{Machine-mode status register ({\tt mstatus}) for RV64 and RV128.}
\label{hypervisor-mstatus}
\end{figure*}

The MPV bit (Machine Previous Virtualization Mode) is written by the implementation
whenever a trap is taken into M-mode.  Just as the MPP bit is set to the privilege
mode at the time of the trap, the MPV bit is set to the value of the virtualization
mode V at the time of the trap.  When an MRET instruction is executed, the
virtualization mode V is set to MPV, unless MPP=3, in which case V remains 0.

The MTL bit (Machine Translation Level), which indicates which address-translation level
caused an access-fault or page-fault exception, is also written by the implementation whenever a trap
is taken into M-mode.
On an access or page fault due to guest physical address translation, MTL is
set to 1.
For any other trap into M-mode, MTL is set to 0.

The SXL field controls the value of XLEN for HS-mode.
The UXL field controls the value of XLEN for VS-mode or U-mode when V=0, or for
VU-mode when V=1.

The TSR and TVM fields only affect execution in HS-mode.

The TW field affects execution in all modes except M-mode.

The hypervisor extension changes the behavior of the the Modify Privilege
field, MPRV.  When MPRV=0, translation and protection behave as normal.  When
MPRV=1, loads and stores are translated and protected as though the current
privilege mode were set to MPP and the current virtualization mode were set to
MPV.  Table~\ref{h-mprv} enumerates the cases.

\begin{table*}[h!]
\begin{center}
\begin{tabular}{|c|c|c||p{4.5in}|}
  \hline
   MPRV & MPV & MPP & Effect \\ \hline \hline
   0    & --  & --  & Normal access; current privilege and virtualization modes apply. \\ \hline
   1    & 0   & 0   & U-level access with HS-level translation and protection only. \\ \hline
   1    & 0   & 1   & HS-level access with HS-level translation and protection only.  \\ \hline
   1    & --  & 3   & M-level access with no translation. \\ \hline
   1    & 1   & 0   & VU-level access with two-level translation and protection. The HS-level MXR bit makes any executable page readable.  The VS-level MXR makes readable those pages marked executable at the VS translation level only if readable at the guest-physical translation level. \\ \hline
   1    & 1   & 1   & VS-level access with two-level translation and protection. The HS-level MXR bit makes any executable page readable.  The VS-level MXR makes readable those pages marked executable at the VS translation level only if readable at the guest-physical translation level.  The VS-level SUM bit applies instead of HS-level SUM. \\ \hline
 \end{tabular}
\end{center}
\caption{Effect on load and store translation and protection under MPRV.  When MPRV=1, MPP$\neq$3, and {\tt hstatus}.SPRV=1, the effective privilege is further modified: {\tt hstatus}.SPV applies instead of MPV, and the HS-level SPP applies instead of MPP.}
\label{h-mprv}
\end{table*}

The {\tt mstatus} register is a superset of the foreground {\tt sstatus}
register;
modifying a field in {\tt sstatus} modifies the homonymous field in {\tt
mstatus}, and vice-versa.

\section{Two-Level Address Translation}
\label{sec:two-level-translation}

Whenever the current virtualization mode V is 1 (and assuming both
{\tt mstatus}.MPRV=0 and {\tt hstatus}.SPRV=0), two-level address translation
and protection is in effect.
For any virtual memory access, the original virtual address is first converted
by VS-level address translation, as controlled by the VS-level {\tt satp}
register, into a {\em guest physical address}.
The guest physical address is then converted by guest physical address
translation, as controlled by the {\tt hgatp} register, into a supervisor
physical address.
Although there is no option to disable two-level address translation when V=1,
either level of translation can be effectively disabled by zeroing the
corresponding {\tt satp} or {\tt hgatp} register.

The VS-level MXR setting, which makes execute-only pages readable, only
overrides VS-level page protection.
Setting MXR at VS-level does not override guest-physical page protections.
Setting MXR at HS-level, however, overrides both VS-level and guest-physical
execute-only permissions.

When V=1, memory accesses that would normally bypass address translation are
subject to guest physical address translation alone.
This includes memory accesses made in support of VS-level address translation,
such as reads and writes of VS-level page tables.

Machine-level physical memory protection applies to supervisor physical
addresses and is in effect regardless of virtualization mode.

\subsection{Guest Physical Address Translation}
\label{sec:guest-addr-translation}

The mapping of guest physical addresses to supervisor physical addresses is
controlled by CSR {\tt hgatp} (Section~\ref{sec:hgatp}).

When the address translation scheme selected by the MODE field of {\tt hgatp}
is Bare, guest physical addresses are equal to supervisor physical addresses
without modification, and no memory protection applies in the trivial
translation of guest physical addresses to supervisor physical addresses.

When {\tt hgatp}.MODE specifies a translation scheme of Sv32x4, Sv39x4, or
Sv48x4, guest physical address translation is a variation on the usual
page-based virtual address translation scheme of Sv32, Sv39, or Sv48,
respectively.
In each case, the size of the incoming address is widened by 2~bits (to 34, 41,
or 50 bits).
To accommodate the 2~extra bits, the root page table (only) is expanded by a
factor of four to be 16~KiB instead of the usual 4~KiB.
Matching its larger size, the root page table also must be aligned to a 16~KiB
boundary instead of the usual 4~KiB page boundary.
Except as noted, all other aspects of Sv32, Sv39, or Sv48 are adopted unchanged
for guest physical address translation.
Non-root page tables and all page table entries (PTEs) have the same formats as
documented in Sections \ref{sec:sv32}, \ref{sec:sv39}, and~\ref{sec:sv48}.

For Sv32x4, an incoming guest physical address is partitioned into a virtual
page number (VPN) and page offset as shown in Figure~\ref{sv32x4va}.
This partitioning is identical to that for an Sv32 virtual address as depicted
in Figure~\ref{sv32va} (page~\pageref{sv32va}), except with 2 more bits at the
high end in VPN[1].
(Note that the fields of a partitioned guest physical address also correspond
one-for-one with the structure that Sv32 assigns to a physical address,
depicted in Figure~\ref{rv32va}.)

\begin{figure*}[h!]
{\footnotesize
\begin{center}
\begin{tabular}{@{}E@{}O@{}E}
\instbitrange{33}{22} &
\instbitrange{21}{12} &
\instbitrange{11}{0} \\
\hline
\multicolumn{1}{|c|}{VPN[1]} &
\multicolumn{1}{c|}{VPN[0]} &
\multicolumn{1}{c|}{page offset} \\
\hline
12 & 10 & 12 \\
\end{tabular}
\end{center}
}
\vspace{-0.1in}
\caption{Sv32x4 virtual address (guest physical address).}
\label{sv32x4va}
\end{figure*}

For Sv39x4, an incoming guest physical address is partitioned as shown in
Figure~\ref{sv39x4va}.
This partitioning is identical to that for an Sv39 virtual address as depicted
in Figure~\ref{sv39va} (page~\pageref{sv39va}), except with 2 more bits at the
high end in VPN[2].
Address bits 63:41 must all be zeros, or else a page-fault exception occurs,
attributed to guest physical address translation.

\begin{figure*}[h!]
{\footnotesize
\begin{center}
\begin{tabular}{@{}E@{}O@{}O@{}O}
\instbitrange{40}{30} &
\instbitrange{29}{21} &
\instbitrange{20}{12} &
\instbitrange{11}{0} \\
\hline
\multicolumn{1}{|c|}{VPN[2]} &
\multicolumn{1}{c|}{VPN[1]} &
\multicolumn{1}{c|}{VPN[0]} &
\multicolumn{1}{c|}{page offset} \\
\hline
11 & 9 & 9 & 12 \\
\end{tabular}
\end{center}
}
\vspace{-0.1in}
\caption{Sv39x4 virtual address (guest physical address).}
\label{sv39x4va}
\end{figure*}

For Sv48x4, an incoming guest physical address is partitioned as shown in
Figure~\ref{sv48x4va}.
This partitioning is identical to that for an Sv48 virtual address as depicted
in Figure~\ref{sv48va} (page~\pageref{sv48va}), except with 2 more bits at the
high end in VPN[3].
Address bits 63:50 must all be zeros, or else a page-fault exception occurs,
attributed to guest physical address translation.

\begin{figure*}[h!]
{\footnotesize
\begin{center}
\begin{tabular}{@{}E@{}O@{}O@{}O@{}O}
\instbitrange{49}{39} &
\instbitrange{38}{30} &
\instbitrange{29}{21} &
\instbitrange{20}{12} &
\instbitrange{11}{0} \\
\hline
\multicolumn{1}{|c|}{VPN[3]} &
\multicolumn{1}{c|}{VPN[2]} &
\multicolumn{1}{c|}{VPN[1]} &
\multicolumn{1}{c|}{VPN[0]} &
\multicolumn{1}{c|}{page offset} \\
\hline
11 & 9 & 9 & 9 & 12 \\
\end{tabular}
\end{center}
}
\vspace{-0.1in}
\caption{Sv48x4 virtual address (guest physical address).}
\label{sv48x4va}
\end{figure*}

\begin{commentary}
The page-based guest physical address translation scheme for RV32, Sv32x4, is
defined to support a 34-bit guest physical address so that an RV32 hypervisor
need not be limited in its ability to virtualize real 32-bit RISC-V machines,
even those with 33-bit or 34-bit physical addresses.
This may include the possibility of a machine virtualizing itself, if it
happens to use 33-bit or 34-bit physical addresses.
Multiplying the size and alignment of the root page table by a factor of four
is the cheapest way to extend Sv32 to cover a 34-bit address.
The possible wastage of 12~KiB for an unnecessarily large root page table is
expected to be of negligible consequence for most (maybe all) real uses.

A consistent ability to virtualize machines having as much as four times the
physical address space as virtual address space is believed to be of some
utility also for RV64.
For a machine supporting 39-bit virtual addresses (Sv39), for example, this
allows the hypervisor extension to support up to a 41-bit physical address
space without either necessitating hardware support for 48-bit virtual
addresses (Sv48) or falling back to emulating the larger address space with
shadow page tables.
\end{commentary}

The conversion of an Sv32x4, Sv39x4, or Sv48x4 guest physical address is
accomplished with the same algorithm used for Sv32, Sv39, or Sv48, as presented
in Section~\ref{sv32algorithm}, except that:
\begin{compactitem}
\item
in step~1, $a = \mbox{{\tt hgatp}.PPN}\times\mbox{PAGESIZE}$;
\item
the current privilege mode is always taken to be U-mode; and
\item
instead of {\tt mstatus}.MXR, the HS-level MXR applies (usually in
{\tt bsstatus}).
\end{compactitem}

For guest physical address translation, all memory accesses (including those
made to access data structures for VS-level address translation) are considered
to be user-level accesses, as though executed in U-mode.
Access type permissions---readable, writable, or executable---are checked
during guest physical address translation the same as for VS-level address
translation.
For a memory access made to support VS-level address translation (such as to
read/write a VS-level page table), permissions are checked as though for a load
or store, not for the original access type.
However, any exception is always reported for the original access type
(instruction, load, or store/AMO).

Access faults and page faults raised by guest physical address translation are
treated as HS-level exceptions for the purpose of exception delegation, so are
not delegated to VS-mode, regardless of the setting of the {\tt hedeleg}
register.

\subsection{Memory-Management Fences}

The behavior of the SFENCE.VMA instruction is affected by the current
virtualization mode V.  When V=0, the virtual-address argument is an HS-level
virtual address, and the ASID argument is an HS-level ASID.
The instruction orders stores only to HS-level address-translation structures
with subsequent HS-level address translations.

When V=1, the virtual-address argument to SFENCE.VMA is a guest virtual
address within the current virtual machine, and the ASID argument is a VS-level
ASID within the current virtual machine.
The current virtual machine is identified by the VMID field of CSR {\tt hgatp},
and the effective ASID can be considered to be the combination of this VMID
with the VS-level ASID.
The SFENCE.VMA instruction orders stores only to the VS-level
address-translation structures with subsequent VS-level address translations
for the same virtual machine, i.e., only when {\tt hgatp}.VMID is the same as
when the SFENCE.VMA executed.

Hypervisor instructions HFENCE.GVMA and HFENCE.BVMA provide additional
memory-management fences to complement SFENCE.VMA.
These instructions are described in Section~\ref{sec:hfence.vma}.

Section~\ref{pmp-vmem} discusses the intersection between physical memory
protection (PMP) and page-based address translation.
It is noted there that, when PMP settings are modified in a manner that affects
either the physical memory that holds page tables or the physical memory to
which page tables point, M-mode software must synchronize the PMP settings with
the virtual memory system.
For HS-level address translation, this is accomplished by executing in M-mode
an SFENCE.VMA instruction with {\em rs1}={\tt x0} and {\em rs2}={\tt x0}, after
the PMP CSRs are written.
If guest physical address translation is in use, synchronization with its data
structures is also needed.
When PMP settings are modified in a manner that affects either the physical
memory that holds guest-physical page tables or the physical memory to which
guest-physical page tables point, an HFENCE.GVMA instruction with
{\em rs1}={\tt x0} and {\em rs2}={\tt x0} must be executed in M-mode after the
PMP CSRs are written.
An HFENCE.BVMA instruction is not required.


\clearpage

\section{Base ISA Control}

The {\tt mstatus} field SXL determines XLEN for HS-mode.

When executing in VS-mode, XLEN is determined by the the UXL field of the
background register {\tt bsstatus}.  Because {\tt bsstatus} is swapped with
{\tt sstatus} when transitioning from VS-mode into HS-mode or M-mode, HS-mode
and M-mode can modify VS-mode's XLEN via the UXL field of the foreground
register {\tt sstatus}.

When executing in U-mode or VU-mode, XLEN is determined by the UXL field of the
foreground register {\tt sstatus}.

\begin{commentary}
HS-mode controls U-mode's XLEN the same way it controls VS-mode's XLEN, via
{\tt sstatus}.UXL.
\end{commentary}

\section{Traps}

The hypervisor extension augments the environment-call exception cause
encoding.  Environment calls from HS-mode use cause 9, whereas environment
calls from VS-mode use cause 10.  Table~\ref{hcauses} lists the
possible M-mode and HS-mode exception codes when the hypervisor extension is
present.

\begin{commentary}
HS-mode and VS-mode ECALLs use different cause values so they can be delegated
separately.
\end{commentary}

\begin{table*}[h!]
\begin{center}
\begin{tabular}{|r|r|l|l|}

  \hline
  Interrupt & Exception Code  & Description \\
  \hline
  1         & 0               & User software interrupt \\
  1         & 1               & Supervisor software interrupt \\
  1         & 2               & {\em Reserved} \\
  1         & 3               & Machine software interrupt \\
  1         & 4               & User timer interrupt \\
  1         & 5               & Supervisor timer interrupt \\
  1         & 6               & {\em Reserved} \\
  1         & 7               & Machine timer interrupt \\
  1         & 8               & User external interrupt \\
  1         & 9               & Supervisor external interrupt \\
  1         & 10              & {\em Reserved} \\
  1         & 11              & Machine external interrupt \\
  1         & $\ge$12         & {\em Reserved} \\ \hline
  0         & 0               & Instruction address misaligned \\
  0         & 1               & Instruction access fault \\
  0         & 2               & Illegal instruction \\
  0         & 3               & Breakpoint \\
  0         & 4               & Load address misaligned \\
  0         & 5               & Load access fault \\
  0         & 6               & Store/AMO address misaligned \\
  0         & 7               & Store/AMO access fault \\
  0         & 8               & Environment call from U-mode or VU-mode \\
  0         & 9               & Environment call from HS-mode \\
  0         & 10              & Environment call from VS-mode \\
  0         & 11              & Environment call from M-mode \\
  0         & 12              & Instruction page fault \\
  0         & 13              & Load page fault \\
  0         & 14              & {\em Reserved} \\
  0         & 15              & Store/AMO page fault \\
  0         & $\ge$16         & {\em Reserved} \\
  \hline
\end{tabular}
\end{center}
\caption{Supervisor and machine cause register ({\tt scause} and {\tt mcause}) values when the hypervisor extension is enabled.}
\label{hcauses}
\end{table*}

When a trap occurs in HS-mode or U-mode, it goes to M-mode, unless
delegated by {\tt medeleg} or {\tt mideleg}, in which case it goes to HS-mode.
If the N extension for user-mode interrupts is implemented, then U-mode
traps destined for HS-mode may be further delegated to U-mode using the {\tt
sedeleg} and {\tt sideleg} CSRs.

When a trap occurs in VS-mode or VU-mode, it goes to M-mode, unless
delegated by {\tt medeleg} or {\tt mideleg}, in which case it goes to HS-mode,
unless further delegated by {\tt hedeleg} or {\tt hideleg}, in which case it
goes to VS-mode.  If the N extension for user-mode interrupts is implemented,
then VU-mode traps destined for VS-mode may be further delegated to VU-mode
using the {\tt sedeleg} and {\tt sideleg} CSRs.

When a trap is taken into M-mode, the following occurs: first, if the
virtualization mode V was 1, the contents of the background supervisor
registers are swapped with their foreground counterparts.  Then, {\tt
mstatus}.MPV and {\tt mstatus}.MPP are set according to Table~\ref{h-mpp}.

\begin{table*}[h!]
\begin{center}
\begin{tabular}{|l|c|c|}
  \hline
  Previous Mode & MPV & MPP \\ \hline
  U-mode        & 0   & 0   \\
  HS-mode       & 0   & 1   \\
  M-mode        & 0   & 3   \\ \hline
  VU-mode       & 1   & 0   \\
  VS-mode       & 1   & 1   \\ \hline
\end{tabular}
\end{center}
\caption{Value of {\tt mstatus} fields MPV and MPP after a trap into M-mode.
Upon trap return, MPV is ignored when MPP=3.}
\label{h-mpp}
\end{table*}

When a trap is taken into HS-mode, the following occurs: first, if the
virtualization mode V was 1, the contents of the background supervisor
registers are swapped with their foreground counterparts.
Then, {\tt hstatus}.SP2V is set to {\tt hstatus}.SPV, {\tt hstatus}.SP2P is set
to {\tt sstatus}.SPP, and lastly {\tt hstatus}.SPV and {\tt sstatus}.SPP are
set according to Table~\ref{h-spp}.

\begin{table*}[h!]
\begin{center}
\begin{tabular}{|l|c|c|}
  \hline
  Previous Mode & SPV & SPP \\ \hline
  U-mode        & 0   & 0   \\
  HS-mode       & 0   & 1   \\ \hline
  VU-mode       & 1   & 0   \\
  VS-mode       & 1   & 1   \\ \hline
\end{tabular}
\end{center}
\caption{Value of {\tt hstatus} field SPV and {\tt sstatus} field SPP after a trap into HS-mode.}
\label{h-spp}
\end{table*}

When a trap is taken into VS-mode, {\tt sstatus}.SPP is set according to
Table~\ref{h-vspp}.
Bits SP2V, SP2P, and SPV of {\tt hstatus} are not modified, and the current
virtualization state V remains 1.

\begin{table*}[h!]
\begin{center}
\begin{tabular}{|l|c|c|}
  \hline
  Previous Mode & SPP \\ \hline
  VU-mode       & 0   \\
  VS-mode       & 1   \\ \hline
\end{tabular}
\end{center}
\caption{Value of {\tt sstatus} field SPP after a trap into VS-mode.}
\label{h-vspp}
\end{table*}

\section{Trap Return}

The MRET instruction is used to return from a trap taken into M-mode.
MRET sets the privilege mode according to the values of MPP and MPV in
{\tt mstatus}, as encoded in Table~\ref{h-mpp}.
MRET then in {\tt mstatus} sets MPV=0, MPP=0, MIE=MPIE, and MPIE=1, and also
sets {\tt pc}={\tt mepc}.
Finally, if the new virtualization mode V=1, the contents of the background
supervisor registers are swapped with their foreground counterparts.

The SRET instruction is used to return from a trap taken into HS-mode or
VS-mode.  Its behavior depends on the current virtualization mode.

When executed in M-mode or HS-mode (i.e., V=0), SRET sets the virtualization
and privilege modes according to the values in {\tt hstatus}.SPV and
{\tt sstatus}.SPP, as encoded in Table~\ref{h-spp}.
SRET then sets {\tt hstatus}.SPV={\tt hstatus}.SP2V,
{\tt sstatus}.SPP={\tt hstatus}.SP2P, {\tt hstatus}.SP2V=0,
{\tt hstatus}.SP2P=0, {\tt sstatus}.SIE={\tt sstatus}.SPIE,
{\tt sstatus}.SPIE=1, and {\tt pc}={\tt sepc}.
Finally, if the new virtualization mode V=1, the contents of the background
supervisor registers are swapped with their foreground counterparts.

When executed in VS-mode (i.e., V=1), SRET sets the privilege mode according to
Table~\ref{h-vspp}, then in {\tt sstatus} sets SPP=0, SIE=SPIE, and SPIE=1, and
lastly sets {\tt pc}={\tt sepc}.
